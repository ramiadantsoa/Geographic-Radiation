\begin{strip}
{\setstretch{1.0}
{\huge \bfseries Phylogenetic Inference of Geographical Radiation} \\ 

Tanjona Ramiadantsoa$^{1*}$, Jukka Sir{\'e}n$^1$ and Ilkka Hanski$^1$ \\

$^1$Metapopulation Research Centre, Department of Biosciences, P.O. Box 65 (Viikinkaari 1), FI -00014 University of Helsinki, Finland \\

*Corresponding author. E-mail: tramiada@umn.edu  \\

Key words: Phylogenetic Comparative Analysis, Radiation, Colonization, Range Dynamics, Approximate Bayesian Computation, Malagasy dung beetles. 
}
\section*{Abstract}
In adaptive radiations, speciation and ecological diversification lead to an increase in the number of species until opportunities for the establishment of further species with sufficiently dissimilar specializations are exhausted. 
Many methods of phylogenetic comparative analysis have been developed to study association between diversification rate and character state, but no previous analysis has examined the coupling between range dynamics and adaptive radiation.  
Here, we construct a continuous-time discrete-space Markov model that couples range dynamics with a model of evolutionary radiation, where speciation or extinction rate is diversity dependent. 
The model is fitted using Approximate Bayesian Computation and tested with simulated data for a system of five interconnected regions. 
We validated the inference by estimating parameter values from simulated data, and found that simultaneous inference of region-specific speciation, extinction and colonization rates is not possible. 
The likely reason is correlations among parameter values, as we are able to estimate the ratios of within-region speciation to extinction rates. 
We applied the model to a monophyletic lineage of 74 species of endemic dung beetles (Canthonini: \textit{Nanos} and \textit{Apotolamprus}) in Madagascar. 
The estimated within-region speciation rates and the between-region colonization rate are of similar magnitude, highlighting the importance of explicitly incorporating range dynamics in a model of radiation. 
The estimated extinction rate is clearly lower in northern than in eastern or western Madagascar, while the estimated within-region speciation rate shows less marked differences between the regions, though is somewhat higher in North. 
The current species richness highest in North with complex topography and a mixture of biomes, which is likely to have buffered against extinctions even during long periods of time with changing climatic conditions. 
The approach we have developed here is a step towards examining the weaknesses and the strengths of phylogenetic comparative methods in the explicitly spatial context. 
Further development of the model is needed before routine application to empirical data.  
\end{strip}
