\section*{Materials and methods}
\subsection*{Model}
\noindent The model, which we call the GR (Geographic-Radiation) model for short, is an extension of the GeoSSE model \citep{Goldberg2011} to multiple regions. 
GR model is a continuous-time discrete-space Markov process, modeling the occupancy state of each region, with no consideration for population sizes in the regions. 
While modeling the dynamics of a lineage, three events may occur: colonization, extinction, and speciation. 
Colonization can only happen between adjacent regions, at a per lineage rate $c$, and it turns an unoccupied region into an occupied region. 
An occupied region may become unoccupied due to an extinction event, which happens in each region at a per lineage rate $e$. 
An extinction event leads to either range contraction, if the lineage occurred in two or more regions, or to global lineage extinction, if the species was endemic to the region. 
We assume throughout this paper that $c$ is the same constant for all regions, while $e$ is either the same constant for all regions or has region-specific values. 
Parameter $c$ is always independent of the current number of lineages in the source and recipient regions, while $e$ may depend on the current number of lineages in the region (below).
%
\begin{figure*}[!t]
\begin{center}
\scalebox{0.5}{\includegraphics{./figs/I_fig1.png}}
\caption{a) The phylogeny and distribution of the Nanos-Apotolamprus clade adapted from \citet{Miraldo2014}. 
		Each color corresponds to a different region in Madagascar – see the map in b). 
		c) The geographical setting of the five regions with their respective connectivity represented by arrows.}
\label{fig:1}
\vspace{-0.3in}
\end{center}
\end{figure*}
%
Speciation may happen in two different ways, either within a region or between regions. 
Within-region speciation refers to a speciation event within a single region. 
Because the regions are large and have heterogeneous environments, within-region speciation does not need to be real sympatric speciation but may be allopatric within the region. 
Within-region speciation event gives rise to two sister species, the first one having the same distribution as the parent species and the second one being endemic to the region in which speciation occurred. 
We consider scenarios in which the within-region speciation rate $s$ is the same constant for all regions or may have region-specific values. 
In between-region (allopatric) speciation, the ancestral species occupies at least two regions, and the ranges of the sister species are obtained by splitting the ancestral species’ range into two disjoint sub-regions. 
If spatial configuration is one-dimensional, as it is here (below), and if the range of the ancestral species consists of $m$ regions, there are $m-1$ possible bipartitions, each of which leads to between-region speciation at a constant rate $a$. 
Hence, the wider the distribution of a lineage, the greater the probability of between-region speciation. 
Finally, to initiate the radiation, we need to specify the initial range $i_R$, which gives the range (a single region) for the common ancestor.
 
The number of extant species in a region can influence the dynamics in two ways and thereby introduces diversity (lineage number) dependence into the model. 
In \textit{Model A}, we assume that within-region speciation rate is diversity-dependent. 
We model this by multiplying the per lineage within-region speciation rate by a decreasing function $f(n)$, where $n$ is the number of lineages in the region. 
In \textit{Model B}, we assume that the number of extant lineages in the region affects the per lineage extinction rate, which is multiplied by an increasing function \textit{g(n)}. 
We assume that $f$ and $g$ have the same functional form for all regions. 
Because one of our aims is to make inference about radiation in Malagasy dung beetles, we assume 5 regions where the classification follows that of  \citep{Miraldo2014}.

\subsection*{Application to Malagasy Dung Beetles}
\noindent We apply the model to the radiation consisting of the genera \textit{Nanos} and \textit{Apotolamprus}, which is known to be monophyletic and colonized Madagascar 24 to 13 million years ago \citep{Miraldo2014}. 
There are altogether 74 known species \citep{Montreuil2014} of which 50 species are included in the molecular phylogeny of \citep{Miraldo2014} (\cref{fig:1}a). 
\citet{Miraldo2014} further showed that there is a decline in the diversification rate toward the present, which makes the data suitable the modeling purpose. 
Madagascar is divided into five geographical regions: North (N), North-East (NE), North-West (NW), South-East (SE), and South-West (SW) (Fig. 1b and 1c). 
The regions are not physically isolated, but they have different climatic and environmental conditions and correspond to a high-level classification of recognized Malagasy biogeographic regions \citep{Wilme2006}. 
There is an especially strong contrast between the humid eastern regions (with rain forest) and dry western regions (various dry forest types and open habitats). 
Considering this contrast, and the fact that the high-altitude plateau in central Madagascar has very few dung beetles, colonization between East (NE and SE) and West (NW and SW) is very unlikely and we set the respective colonization rates to 0. 
The spatial configuration of the five regions is thereby reduced to an one-dimensional space (Fig. 1c). Below, when we allow for variation in speciation and extinction rates among the regions, we assume that the two eastern and the two western regions share the same values, thus there may be three speciation $(s_N,s_W,s_E )$ and extinction $(e_N,e_W,e_E)$ rates.

\subsection*{Statistical Model}
\noindent Our goal is to infer the distribution of the model parameters that would produce the empirically observed phylogeny and the pattern of geographical ranges of the species (Fig. 1a). 
We consider four different scenarios to explore the power of the model to infer the model parameter. 
In Scenario 1, we attempt to estimate both region-specific speciation and extinction rates, and thus the vector of parameter values is given by $\varphi_1=(s_N,s_W, s_E, e_N,e_W,e_E,c,a,i_R)$. 
In Scenario 2, we fix one of the extinction rates to 1 to set the time scale, and estimate the vector  $\varphi_2=(s_N,s_W, s_E, e_W,e_E,c,a,i_R)$.
In Scenario 3, we further simplify the model and assume that all extinction rate parameters have the same value $(e_. = 1)$, thus $\varphi_3=(s_N,s_W,s_E, c,a,i_R)$. 
Finally, in Scenario 4, the within-region speciation rate parameters, instead of the extinction rate parameters, have the same value $(s_. = 1)$ in each region, and thus $\varphi_4=(e_N,e_W,e_E,c,a,i_R)$. 

The closely-related GeoSSE model \citep{Goldberg2011} assumes two regions, for which the likelihood function is constructed and solved. 
This leads to a set of three differential equations along each branch of the tree. 
In this approach, the number of differential equations increases exponentially with the number of regions, and becomes even more complicated if the model includes diversity-dependent parameters. 
Therefore, instead of constructing and solving the likelihood function required by traditional Bayesian computational methods \citep{Robert2013}, we employ the Approximate Bayesian Computation (ABC) to estimate the posterior probability density of the parameters. 
In the application to Malagasy dung beetles below, we assume uninformative uniform prior distributions for $\varphi$, with ranges $[0,10]$ for the extinction and within-region speciation rate parameters, $[0,5]$ for the between-region speciation rate parameter and $[0,2]$ for the colonization rate parameter. 

ABC is a flexible family of methods based on simulating an appropriate model. 
In the last decade, ABC has proliferated due to increasing complexity of data and models that researchers study \citep{Toni2009, Beaumont2010, Sunnaker2013}. 
ABC is a helpful method when the likelihood function is intractable or is too costly to compute (which is the case here). 
There are several versions of ABC, but the main idea is as follows: 1. Generate a vector $\varphi$ of parameter values from a distribution, usually the prior distribution. 
2. Simulate the mathematical model using $\varphi$ to generate a simulated data set.
3. Compare the simulated and empirical data sets. 
If the distance between the two data sets is smaller than a fixed tolerance $\varepsilon$, the proposed set of parameters is accepted. 
If these steps are repeated many times and the tolerance $\varepsilon$ is small enough, meaning that we only accept parameters that produce a data set that is very similar to the empirical one, the set of accepted parameter values provides a good approximation of the true unknown posterior distribution of the model parameter. 
In the rest of this article, we use interchangeably posterior distribution and approximate posterior distribution.  
‘Distance’ is interpreted liberally, and if the data cannot be compared directly, a set of summary statistics is used to capture the essential features of the data. 
The summary statistics project the data into a low-dimensional space and introduce another level of approximation to the method. 
Nonetheless, with sufficient summary statistics, the distribution targeted by the ABC algorithm matches the posterior distribution, but such choices are generally unavailable outside simple models. 
In general, the summary statistics have to be constructed by balancing the amount of information they carry with low dimensionality to keep the computational complexity of the algorithm feasible. 
The choice of summary statistics is therefore a crucial component of ABC, and we next describe the summary statistics used here, followed by a detailed description of the implementation of the ABC.

\subsection*{Summary Statistics}
\noindent As explained above, summary statistics are the key component of ABC. 
We tested several statistics such as tree imbalance described by the Colless’ index \citep{Colless1982}, the distribution of relatednesses (i.e. how many internal nodes need to be crossed to move from one tip to another, with sister species having relatedness 1, cousins relatedness 3), and the Lineage-Through-Time (LTT) plot, but none of these were informative. 
At the end, we retained four summary statistics: the number of species per region, the distribution of species’ range sizes, the distribution of range similarity between pairs of sister species, and the relative phylogenetic diversity. 
Moreover, it is essential to compare trees of the same size (the same number of tips). 
Since it is very likely that the number of species obtained at the end of each simulation will be different from the number of species in the empirical tree, denoted by $n_e$, we randomly sample $n_e$ species from the simulated data and prune the resulting tree accordingly. 
The length of the simulation is described below. 
The distance between the simulated and empirical trees, denoted by $A$ and $\hat{A}$ respectively, is computed using the product of the four summary statistics.

In the case of the number of species per region, we calculate the Euclidian distance between the two trees as
\begin{equation}
	d^1= \sqrt{\sum_{i \in \{N,NE,NW,SW,SE\}}[ s(A ,i)-s(\hat{A},i)]^2 }
\label{eq:d1}
\end{equation}
where $s(.,i)$ counts the number of species in region $i$. 
The second statistic $d^2$ compares the distributions of range sizes, and is obtained by computing the difference between the number of species occupying $1,2, \ldots, 5$ regions, $r(.,i)$, as
\begin{equation}
	d^2= \sqrt{\sum_{i \in [\![1,5]\!]}[r(A ,i)-r(\hat{A},i)]^2 }
\label{eq:d2}
\end{equation}
 The third statistic measures the similarity in the distributions of sister species. 
 We consider three categories of distributions. 
 If the sister species have exactly the same distributions their pairwise distance is 0.
 If the distribution of one species is nested within the distribution of the other one the distance is 1, and otherwise the distance is 2.  
 Our third summary statistic is then given by 
 \begin{equation}
	d^3= \sqrt{\sum_{k = 0}^2[ u(A ,k)-u(\hat{A},k)]^2 }
\label{eq:d3}
\end{equation}
Here $u(.,k)$ computes the proportion of sister species pairs that have the same distributions $(k=0)$ have nested distributions $(k=1)$, and other types of distributions $(k=2)$. 
Finally, phylogenetic distance was calculated by summing up the total length of the branches, starting from the crown age, divided by the age of the crown. 
This function, denoted by $\nu(.)$, allows one to characterize the relative phylogenetic diversity of any tree, regardless of the length of the simulation. 
The fourth distance is given by
\begin{equation}
	d^4=|\nu(A)- \nu(\hat{A})|.	
\label{eq:d4}
\end{equation}
 %
\begin{table*}[b]
\caption{Summary of different scenarios and model assumptions.} \label{tab:1}
%\begin{center}
\centering
\begin{tabular}{cc}
\hline
\textit{Model A} & \textit{Model B} \\
Within-region speciation rate is diversity-dependent &  Extinction rate is diversity-dependent \\ 
\hline
\multicolumn{2}{c}{Scenario 1: all parameters are estimated}  \\
\multicolumn{2}{c}{Scenario 2: $e_N = 1$ and is not estimated} \\
\multicolumn{2}{c}{Scenario 3: $e_W = e_N = e_E = 1$ and are not estimated}\\
\multicolumn{2}{c}{Scenario 4: $s_W = s_N = s_E = 1$ and are not estimated} \\
\hline
\end{tabular}
%\end{center}
\end{table*}
%
\subsection*{Implementation of ABC}
\noindent We use a sequential Monte Carlo ABC combined with adaptive tuning of tolerance and adjustment of the perturbation kernel, comparable to the framework used by \citet{Numminen2013}. 
In a sequential ABC, instead of using one value for tolerance and one round of rejecting samples, a sequence of intermediate distributions with decreasing tolerances are utilized. 
Each generation is terminated when a required number of parameter values $N$ has been accepted. 
The set of accepted parameters $\{\varphi_i \}_{i=1,\ldots,N}$ and their associated weights $\{w_i \}_{i=1, \ldots, N}$ represent a sample from the intermediate distribution, which is used as a proposal distribution for sampling in the next generation. 
To ensure that there remains variation among the parameters in successive generations, they are perturbed according to a perturbation kernel. 
Because we have both continuous and categorical variables to estimate, the perturbation kernel $K$ is a Cartesian product of a multivariate normal distribution $K_1$ and a circular kernel $K_2$.  
The variance-covariance matrix $\Sigma$ for kernel $K_1$ is computed based on the sample from the previous generation
\begin{equation}
	\Sigma_{i,j} = \sum_{k=1}^N \sum_{l=1}^N w_k w_l  (\varphi_l^i - \varphi_k^i)  (\varphi_l^j- \varphi_k^j) 
\label{eq:varcovar}
\end{equation}
Kernel $K_2$ is used for the initial range and is assumed to be circular on the set \{SW,NW,N,NE,E\} with a probability $p/2$ to move to the right, $p/2$ to move to the left, and $1-p$ to stay in the same region. 

The vector of tolerance values $\varepsilon=(\varepsilon^1,\varepsilon^2,\varepsilon^3,\varepsilon^4)$ is updated in the following manner. 
If $\{(d_i^1,d_i^2,d_i^3,d_i^4 )\}_{i=1,\ldots,N}$ is the distance between the simulated and empirical data for the accepted parameter vector $\{\varphi_i \}_{i=1,\ldots,N}$, then $\varepsilon^j$ for the next generation is chosen as the quantile of $d_.^j$ such that a fixed percentage of  $d_.^j$ is accepted. 
This provides a robust way of combining information from the different statistics and accounts for possible correlations between the statistics.
 
The exact algorithm is described as follows:
A. First generation
\begin{enumerate} 
	\item Sample a proposed parameter vector $\varphi$ from the prior distribution $\pi$
	\item Simulate a tree using parameter values $\varphi$ and the GR model (for the length of the simulation see below)
	\item If the number of species is less than $n_{min}$  or more than $n_{max}$ go to step 1, otherwise accept $\varphi$
	\item Sample $n_e$  species from the simulated tree and compute distance $d=(d^1,d^2,d^3,d^4)$ between simulated and empirical trees
	\item If the number of accepted parameter values (=size of the sample) is smaller than $N$, go to step 1, otherwise set $w_i=1/N ,i=1, \ldots ,N$ and go to B. \newline
\end{enumerate}
%
B. Subsequent generations. 
At this stage, we have a set  $ \{\varphi_i,w_i,d_i\}_{i=1,\ldots,N}$ representing the set of accepted parameters, their associated weights, and the distance between the simulated and empirical data.  
The next steps are:
%\end{minipage}
%}
\begin{enumerate} 
	\item Compute the variance-covariance matrix $\Sigma$ 
	\item Initialize the tolerance $\varepsilon=(\varepsilon^1,\varepsilon^2,\varepsilon^3,\varepsilon^4)$
	\item Sample $\varphi^* \mbox{ from } \{\varphi_i\}_{i=1,\ldots,N}$ with respect to $\{w_i \}_{i=1,\ldots,N}$  and perturb such that $\varphi \sim K(\varphi^*,\Sigma,p)$
	\item Simulate tree using the perturbed values $\varphi$ and the GR model
	\item If the number of species is less than $n_{min}$ or more than $n_{max}$ go to step 3
	\item Sample $n_e$ species from the simulated tree and compute distance $d$ between simulated and empirical trees
	\item If the distance $d^i \leq \varepsilon^i$ for each $i=1,\ldots,4$ accept $\varphi$ 
	\item If the number of accepted parameter values is smaller than $N$, go to step 3, otherwise go to C.
\end{enumerate}
C. For each $k=1,\ldots,N$, compute 
\[
	w_k  \propto \frac{ \pi (\varphi_k)}{\sum_{(n=1)}^N \tilde{w_n}  K( \tilde{\varphi_n}, \varphi_k,\Sigma,p)}
\]
and normalize such that  $\sum_{k = 1}^N w_k=1$. 
Here $\{\tilde{\varphi_.},\tilde{w_.}\}$ denotes the vector of parameters and weights from the previous generation (i.e. the proposal distribution), whereas $\varphi_.$ is the newly accepted parameter vector. 
Return to B if the resulting distribution has not converged. 
For convergence, we rely on visual inspection of the marginal posterior distributions.

\subsection*{Model Simulations}
\noindent The simulation of the GR model is based on the Gillespie algorithm; the simulation and the ABC algorithm are implemented in Mathematica 9.0 \citep{Wolfram2013}. 
We initialize the simulation with one species occurring in a single geographic region, and the simulation is run until time $T$. 
Because our model assumes diversity dependence, the number of species fluctuates around a quasi-stationary equilibrium following the initial transient. 
We select $T$ to represent either an early phase of the radiation, when the quasi-stationary equilibrium has been recently reached, or a late phase, when the number of species has already fluctuated around the expected value for a long time. 
In practice, we choose the longer time to be twice as long as the short one. 
Moreover, the simulation was terminated and the parameter vector was rejected if all species went extinct or the number of species was not within the required limits. 

As explained above, there are 74 described species in the \textit{Nanos-Apotolamprus} radiation, but only 50 species were available for the molecular phylogeny \citep{Miraldo2014}. Additionally, we excluded one of the 74 species because it is known from a single specimen from a single locality. 
We require that the number of species at the end of the simulation is greater than $n_e=73$ and smaller than $n_{max}$. 
If this condition is met, we obtain two samples from the simulation. 
First, we sample 73 species from the simulation and compute $d^1$ and $d^2$ , after which we sample 50 species from the sample of 73 species and compute $d^3$ and $d^4$. 
For the first generation we accept the parameter vector as long as the number of species is between $n_e=73$ and $n_{max}=150$.

For the density-dependent \textit{Model A}, we use 
\[
	f(n)=
	\begin{cases}
		0 &\mbox{ if } n=0, \\
		10/n &\mbox{otherwise},
		\end{cases}
\]
and for \textit{Model B}
\[
	g(n)= \frac{2}{1+ 0.1 \exp(- 0.2 n+6)}.
\]
Table 1 summarized the different models and scenarios we tested.
%
\begin{figure*}[t]
\begin{center}
\scalebox{0.8}{\includegraphics{./figs/I_fig2}}
\caption{Inference validation in Scenario 1 for \textit{Model B}.
	The panels compare the true value (x-axis) with the mean of the marginal approximate posterior distribution or with the mean 		square error (y-axis).
 	Estimation of within-region speciation and extinction are denoted by a dot and cross, respectively. 
 	The line represents the identity line and the curve the mean square error under the prior. 
 	The last two row shows estimates of the ratio speciation/extinction for each region. Simulation parameters: $T = 8, N = 400$ 			and $q = 50\%$.}
\label{fig:2}
\vspace{-0.3in}
\end{center}
\end{figure*}

\subsection*{Validation of Model Inference}
\noindent We validated our approach to parameter estimation by applying the statistical model to simulated data.
For this purpose, we selected the four scenarios described in the previous section to simulate, and run simulations for each combination of simulation time $T = 8$ and 16 and \textit{Models A} and \textit{B}. 
For Scenarios 1, 2 and 4, we generated 10 random vectors of parameter values $\varphi$ based on the priors used in the empirical case. 
For Scenario 3, we used 20 random vectors.  
However, we avoided parameters that are close to the upper boundary of the prior distribution, as the support of the posterior distribution of extreme values will not be within the support of the prior distribution. 
In practice, we generated parameters within the lower three fourths of the range of the prior. 
We selected parameters that resulted in at least 50 species and at most of 150 species. 
Moreover, we only accepted parameter values that predicted a similar number of species in at least 3 out of 5 replicates, and further required that the difference between the replicates in the number of species was less than 20. 
The tree and the number of species $n_s$  used for the estimation were obtained from the first simulation. 
We did not perform any sampling unlike in the case of the empirical data. 
We then estimated each set of parameters using the ABC framework described above, requiring that $n_{min}=n_s$  and $n_{max}=n_s+50$. 
We estimated each parameter using the same simulation time $T$ that was used to generate the data. 
We computed the mean square error for each vector of parameters and compared the true value with the mean estimate. 
For the discrete variable of initial range, we computed the probability of having the correct value. 


