\section*{Results}
\subsection*{Validation of Model Inference}
\noindent In the first scenario, we attempted to estimate a model in which there are three region-specific extinction rate parameters, $e_W,e_E$  and $e_N$, and three within-region speciation rate parameters, $s_W,s_E$  and $s_N$.  
Since \textit{Models A} and \textit{B} gave similar results, we only present the latter here (full results are reported in Supplementary file S1). 
In this scenario, the model has low power in inferring the true parameter values and systematically overestimates small values (Fig. 2). 
Relatively poor inference is due to strong correlations between extinction and speciation rate parameters for particular regions (Supplementary file S1). 
However, ratio of within-region speciation to extinction rates can be inferred more successfully (Fig. 2: bottom row). 
In both models, the next strongest correlations, following the ones between within-region speciation and extinction rates, were between colonization and extinction rates (Supplementary file S1). 
Scenario 2 produced essentially the same results, which are given in full in Supplementary file S2.

In Scenario 3, all regions have the same fixed extinction rate, $e = 1$. 
The results are given for $T=8$ in Figure 3; the results were qualitatively similar for other values of $T$ (Supplementary file S3), with some quantitative differences (below).  
Each panel in Figure 3 shows either the mean of the posterior distribution against the true value or the mean square error against the true value. 
In general, the model performs well and the data points lie around the identity line (Fig. 3). 
Overestimation is most apparent for small values of the parameters, for the rate of sympatric speciation in North, and for between-region speciation. 
In almost all cases, the mean square error lies well under the mean square error of the prior (Fig. 3, Supplementary file S3). 
Because the system is symmetric with respect to North, there is no difference between the estimates for $s_W$ and $s_E$ (Fig. 3). 
Irrespective of the time $T$ and the model, the colonization rate is the best-estimated parameter (Fig. 3). 
%
\begin{figure*}[t]
\begin{center}
\scalebox{0.7}{\includegraphics[trim= {0.35in 0 0.2in 0}, clip]{./figs/I_fig3}}
\caption{Inference validation in Scenario 3. 
	Columns from left to right represent respectively: within-region speciation rate in West ($s_W$), North ($s_N$) and East 				($s_E$), colonization rate ($c$), and between-region speciation rate ($a$). 
	Rows 1 and 3 compare the true values (x-axis) and the mean estimates (y-axis) for Models A and B respectively. 
	Rows 2 and 4 shows the mean square error (y-axis) in estimating parameter (x-axis) for Models A and B. 
	Lines represent the identity line, and curves the mean square error of the prior. Simulation parameters: $T = 8, N = 400$ and $q = 50\%$.}
\label{fig:3}
\vspace{-0.2in}
\end{center}
\end{figure*}
 
Inferring the region of the initial colonization, by the last common ancestor, turned out to be difficult. 
Generally, the probability of correctly estimating the initial region is around 0.2 (Supplementary file S3), which is the probability of randomly choosing a region out of 5 regions. 
However, for some parameter values the probability of correctly estimating the initial region was as high as 0.4 in \textit{Model A} and 0.7 in \textit{Model B}. 
This happened when the initial region was either $SW$ or $SE$, the two most peripheral regions (Fig. 1c).   

The estimated parameter values tend to be smaller the longer the simulation time $T$ (Supplementary file S3). 
This is because for long simulations, larger values of parameters tend to produce a larger number of species. 
In general, whether we compare the mean of the posterior distribution with the true value or the mean square error with its true value, the estimation produced somewhat better results for larger values of $T$ (Fig. 3 and Supplementary file S3). 
The between-region speciation rate is the only parameter that is not dependent on $T$ (Fig. 3 and Supplementary file S3), but it is generally overestimated except for large values.  Additionally, the variance of the approximate posterior distribution is independent of the parameter value and decreases with increasing time (Supplementary file S3), indicating that larger values of $T$ lead to more accurate estimates. 

Scenario 4, in which within-region speciation rate is fixed to the same constant value of 1 in each region, while the extinction rate may have different values for the regions, produced comparable results to those from Scenario 3. 
The results are reported in Supplementary file S4.
%
\begin{figure*}[t]
\begin{center}
\scalebox{0.95}{\includegraphics[trim = {0.4in 0 0.2in 0}, clip]{./figs/I_fig4}}
\caption{Approximate posterior distributions for the empirical data in \textit{Model B} Scenario 2. 
	Parameter values: $T = 8, N = 1000$ and $q = 50\%$.}
\label{fig:4}
\vspace{-0.2in}
\end{center}
\end{figure*}

For Scenario 3, we performed an additional test to assess the power of the estimation with respect to the size of the tree. 
The results are given in Supplementary file S5. 
The general conclusion is that the scaled mean square error, defined as the mean square error divided by the true parameter value, decreases with increasing number of species. 
Thus, the parameters are better estimated for larger trees, though there were some exceptions in our particular case, concerning the within-region speciation rate in North and the between-region speciation rate. 
These exceptions were observed for low within-region speciation rate in North in comparison with the within-region speciation rates in East and West. 

\subsection*{Parameter Estimation from Empirical Data}
\noindent Though parameter estimation for the model with both region-specific extinction and speciation rates was not very successful with simulated data, we nonetheless fitted this model to the empirical data, with the extinction rate in North fixed to one (Scenario 2). 
Figure 4 shows the posterior distributions for the parameter values in \textit{Model B} (qualitatively similar results for \textit{Model A}, full results in Supplementary materials S1 and S2). 
In general, the posterior distributions are not very informative, but the following is worth noting. 
The mode of the posterior distributions for within-region speciation rates are about the same in all regions, though the distribution is much narrower in North than in West and East. 
The colonization rate is of the same magnitude as the within-region speciation rates.  
Extinction rate is highest in West and clearly lowest in North ($e_N=1$ by assumption).  
%
\begin{figure*}[t]
\begin{center}
\scalebox{0.9}{\includegraphics[trim = {0.35in 0 0.2in 0}, clip]{./figs/I_fig5}}
\caption{Approximate posterior distributions for the empirical data for Scenario 4. 
	Columns from left to right represent, respectively: extinction rate in West ($e_W$), North ($e_N$) and East ($e_E$), colonization rate ($c$), and between-region speciation rate ($a$).
	The distributions are from the $7^th$ generation, gray and black curve are for $T = 8$ and $T = 16$, respectively. 
	Upper (resp. lower) row is for \textit{Model A} (resp. \textit{B}). 
	Parameters: $q = 50\%$ and sample size $N = 1000$.}
\label{fig:5}
\vspace{-0.2in}
\end{center}
\end{figure*}
%
We next simplify the model by assuming that either the rate of within-region speciation $s$ (Scenario 4) or the extinction rate $e$ (Scenario 3) is the same constant in all regions. 
In Figure 4, the mode of the distributions for within-region speciation rate is about the same, hence we start with Scenario 4 in which $s$ is fixed to 1 in all regions. 
Figure 5 shows the approximate posterior distribution after 7 generations for each parameter. 
The approximate posterior distributions do not differ much for $T= 8$ and 16. 
The results are consistent with the results of the full model in Figure 4. 
The rate of extinction is clearly lowest in North. 
The estimates of the extinction rates are broad in East and West, as they are also in Figure 4, but the former has a somewhat thinner tail than the latter.

We then run the estimation for Scenario 3, in which we assume the same constant extinction rate $e=1$ in all regions. 
Figure 6 shows the approximate posterior distribution after 7 generations, and the full results are given in Supplementary file S3. 
The parameter estimates do not depend much on the value of $T$, nor on sample size $N$ and the values of $q$ and $n_{max}$, though smaller values of  $n_{max}$  (100 instead of 150) lead to slightly lower estimates of within-region speciation rate for all regions. 
The rate of between-region speciation has a mode close to zero while the bulk of the distribution lies between 0 and 2 and is the same for all models and simulation times. 
This parameter is poorly estimated in both Scenarios 3 and 4 as well as in the simulated data (Fig. 3). 
In contrast, the rates of within-region speciation are well estimated, especially for West and East, and the approximate posterior distribution for the colonization rate is also narrow. Compared with the full model in which both within-region speciation and extinction rates are region-specific (Fig. 4), the posterior distributions are here (Fig. 6) much narrower, but the differences between the regions are the same: within-region speciation rate is higher in North than in East and West, and the colonization rate is of the same magnitude as the speciation rate.

Comparing \textit{Model A} and \textit{Model B}, the former yields higher within-region speciation rate in North but lower colonization rate than \textit{Model B}. 
In \textit{Model A}, considering that we have fixed the extinction rate to one (Scenario 3), the results indicate that colonization happens at the lower rate (mean around 0.4 in \textit{Model A}) than regional extinction.
In Scenario 4, where the within-region speciation rate is fixed to 1 in all regions, colonization rate is higher than the extinction rate in North but lower than extinction rate in East and West (Fig. 5). 
In \textit{Model B}, such a comparison is not straightforward because the extinction rate depends on regional species number, but the colonization rate is similar in magnitude to the within-region speciation rate.
Estimating the initial region $i_R$  is clearly difficult as the posterior distribution resembles a uniform distribution, irrespective of the Scenario (3 or 4), the model type and the length of the simulation (see Supplementary files S3 and S4 for complete results). 
This conclusion parallels the result for simulated data (Fig. 3). 
The empirical data suggests rather strongly that the initial colonization occurred in North \citep{Miraldo2014}. We hence repeated the estimation of the other parameter values after fixing the initial range to North. We found no substantial differences between the estimates for the other parameters whether the initial range was fixed to North (or any other region) or not. The complete results are shown in Supplementary files S3 and S4.   
%
\begin{figure*}[t]
\begin{center}
\scalebox{0.9}{\includegraphics[trim = {0.2in 0 0.2in 0}, clip]{./figs/I_fig6}}
\caption{Approximate posterior distributions for the empirical data in Scenario 3. 
	Columns from left to right represent respectively: within-region speciation rate in West ($s_W$), North ($s_N$) and East ($s_E$), colonization rate ($c$), and between-region speciation rate ($a$).
	The distributions are from the $7^th$ generation, gray and black curve are for $T = 8$ and $T = 16$, respectively. 
	Upper (resp. lower) row is for \textit{Model A} (resp. \textit{B}). 
	Parameters: $q = 50\%$ and sample size $N = 1000$. }
\label{fig:6}
\vspace{-0.2in}
\end{center}
\end{figure*}
